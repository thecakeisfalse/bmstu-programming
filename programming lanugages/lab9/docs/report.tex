\documentclass{../../iu9lab}

\labnumber{9}
\course{Языки и методы программирования}
\title{Перегрузка операций}
\group{ИУ9-21Б}
\teacher{Посевин Д. П.}

\begin{document}
    \maketitle
    
    \section{Цель}

    Данная работа предназначена для изучения возможностей языка C++, обеспечивающих применение знаков операций к объектам пользовательских типов.

    \section{Персональный вариант}

    Matrix<T, N> – антидиагональная матрица размера N × N с элементами типа T. (Все элементы антидиагональной матрицы,
    кроме лежащих на диагонали, идущей от нижнего левого угла до верхнего правого угла, равны нулю. Матрица должна быть представлена только числами, лежащими на диагонали)
    Операции:
    \begin{enumerate}
        \item <<+>> -- сумма двух матриц;
        \item <<*>> -- произведение двух матриц;
        \item <<[ ]>> -- получение значения элемента, расположенного на $i$-той строке в $j$-том столбце.
    \end{enumerate}

    \section{Решение}

    \subsection{Код}

    \begin{code}
        \inputminted{cpp}{../src/main.cc}
        \caption{main.cc}
    \end{code}

    \subsection{Скриншоты}

    \begin{figure}[!htbp]
        \centering
        \includegraphics[width=\textwidth]{terminal}
        \caption{Пример работы программы}
    \end{figure}

\end{document}
