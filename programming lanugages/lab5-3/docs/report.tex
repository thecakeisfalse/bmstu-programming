\documentclass{../../iu9lab}

\labnumber{5-3}
\course{Языки и методы программирования}
\title{Монады в языке Java}
\group{ИУ9-21Б}
\teacher{Посевин Д. П.}

\begin{document}
    \maketitle
    
    \section{Цель}

    Приобретение навыков использования монад Optional и Stream в программах на языке Java.

    \section{Персональный вариант}

    Полином с вещественными коэффициентами, для которого определены операции:
    \begin{enumerate}
        \item порождение потока значений полинома в точках, полученных делением отрезка $[a, b]$ на $n$ равных частей;
        \item вычисление определённого интеграла полинома на отрезке $[a, b]$ (если $b < a$, то будем считать, что интеграл не определён).
    \end{enumerate}
    Проверить работу первой операции нужно путём подсчёта сумм отрицательных и неотрицательных значений полинома в потоке.

    \section{Решение}

    \subsection{Код}

    \begin{code}
        \inputminted{java}{../src/Test.java}
        \caption{Test.java}
    \end{code}

    \begin{code}
        \inputminted{java}{../src/Point.java}
        \caption{Point.java}
    \end{code}

    \begin{code}
        \inputminted{java}{../src/Quad.java}
        \caption{Quad.java}
    \end{code}

    \begin{code}
        \inputminted{java}{../src/QuadSet.java}
        \caption{QuadSet.java}
    \end{code}

    \subsection{Скриншоты}

    \begin{figure}[!htbp]
        \centering
        \includegraphics[width=\textwidth]{terminal}
        \caption{Пример работы программы}
    \end{figure}

\end{document}
