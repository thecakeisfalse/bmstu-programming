\documentclass{../../iu9lab}

\labnumber{2}
\course{Языки и методы программирования}
\title{Разработка простейшего класса на языке Java}
\group{ИУ9-21Б}
\teacher{Посевин Д. П.}

\begin{document}
    \maketitle
    
    \section{Цель}

    Целью данной работы является изучение базовых возможностей языка Java.

    \section{Персональный вариант}

    Класс бинарных отношений на множестве целых чисел от 0 до n с двумя операциями: проверка принадлежности пары чисел отношению; вычисление транзитивного замыкания отношения.

    \section{Решение}

    \subsection{Код}

    \begin{code}
        \inputminted{java}{../src/Test.java}
        \caption{Test.java}
    \end{code}

    \begin{code}
        \inputminted{java}{../src/BinaryRelation.java}
        \caption{BinaryRelation.java}
    \end{code}

    \subsection{Скриншоты}

    \begin{figure}[!htbp]
        \centering
        \includegraphics[width=\textwidth]{terminal}
        \caption{Пример работы}
    \end{figure}

    \begin{figure}[!htbp]
        \centering
        \includegraphics[width=\textwidth]{intellij1}
        \includegraphics[width=\textwidth]{intellij2}
        \caption{Исходный код}
    \end{figure}

\end{document}
