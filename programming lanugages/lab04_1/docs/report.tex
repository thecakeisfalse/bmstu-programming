\documentclass{../../iu9lab}

\labnumber{4-1}
\course{Языки и методы программирования}
\title{Реализация итераторов в языке Java}
\group{ИУ9-21Б}
\teacher{Посевин Д. П.}

\begin{document}
    \maketitle
    
    \section{Цель}

    Изучение обобщённых итераторов и экземплярных вложенных классов языка Java.

    \section{Персональный вариант}

    Множество дробей с итератором по всем парам дробей, произведение которых равно $1$.

    \section{Решение}

    \subsection{Код}

    \begin{code}
        \inputminted{java}{../src/Test.java}
        \caption{Test.java}
    \end{code}

    \begin{code}
        \inputminted{java}{../src/Pair.java}
        \caption{Pair.java}
    \end{code}

    \begin{code}
        \inputminted{java}{../src/Fraction.java}
        \caption{Fraction.java}
    \end{code}

    \begin{code}
        \inputminted{java}{../src/FractionsSet.java}
        \caption{FractionsSet.java}
    \end{code}

    \subsection{Скриншоты}

    \begin{figure}[!htbp]
        \centering
        \includegraphics[width=\textwidth]{terminal}
        \caption{Пример работы}
    \end{figure}

\end{document}
