\documentclass{../../iu9lab}

\labnumber{7}
\course{Языки и методы программирования}
\title{Разработка простейшего класса на C++}
\group{ИУ9-21Б}
\teacher{Посевин Д. П.}

\begin{document}
    \maketitle
    
    \section{Цель}

    Целью данной работы является изучение объектно-ориентированных возможностей языка C++.

    \section{Персональный вариант}

    Матрица смежности для простого орграфа с операциями:
    \begin{enumerate}
        \itemsep -0.8pt
        \item определение, ведёт ли дуга из $i$-той вершины в $j$-тую (операция индексации)
        \item добавление вершины;
        \item удаление вершины;
        \item определение, содержит ли граф циклы.
    \end{enumerate}

    \section{Решение}

    \subsection{Код}

    \begin{code}
        \inputminted{cpp}{../src/main.cc}
        \caption{main.cc}
    \end{code}

    \subsection{Скриншоты}

    \begin{figure}[!htbp]
        \centering
        \includegraphics[width=\textwidth]{terminal}
        \caption{Пример работы программы}
    \end{figure}

    \begin{figure}[!htbp]
        \centering
        \includegraphics[width=\textwidth]{graph1}
        \caption{Граф до удаления ребра}
    \end{figure}

    \begin{figure}[!htbp]
        \centering
        \includegraphics[width=\textwidth]{graph2}
        \caption{Граф после удаления ребра}
    \end{figure}

\end{document}
