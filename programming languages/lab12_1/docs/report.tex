\documentclass{../../iu9lab}

\labnumber{12-1}
\course{Языки и методы программирования}
\title{Обработка текстовых файлов}
\group{ИУ9-21Б}
\teacher{Посевин Д. П.}

\begin{document}
    \maketitle
    
    \section{Цель}

    Целью лабораторной работы является приобретение навыка разработки на языке C++ программ, осуществляющих анализ и преобразование текстовых файлов, записанных в различных форматах.

    \section{Персональный вариант}

    Найти все файлы с расширением <<txt>> в указанном каталоге, разбить текст из каждого файла на слова и сформировать в текущем каталоге два файла: all.txt и shared.txt, содержащие объединение и пересечение множеств слов из найденных файлов. Каждое слово в сформированных файлах должно располагаться в отдельной строке. Слова должны быть отсортированы лексикографически. Работоспособность программы нужно проверить на наборе текстовых файлов, содержащих текст на английском языке.

    \section{Решение}

    \subsection{Код}

    \begin{code}
        \inputminted{cpp}{../src/main.cc}
        \caption{main.cc}
    \end{code}

    \subsection{Скриншоты}

    \begin{figure}[!htbp]
        \centering
        \includegraphics[width=\textwidth]{terminal}
        \caption{Пример работы программы}
    \end{figure}

\end{document}
