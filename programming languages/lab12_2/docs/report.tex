\documentclass{../../iu9lab}

\labnumber{12-2}
\course{Языки и методы программирования}
\title{Обработка текстовых файлов}
\group{ИУ9-21Б}
\teacher{Посевин Д. П.}

\begin{document}
    \maketitle
    
    \section{Цель}

    Целью лабораторной работы является приобретение навыка разработки на языке C++ программ, осуществляющих анализ и преобразование текстовых файлов, записанных в различных форматах.

    \section{Персональный вариант}

    Найти все файлы с расширением «svg» в указанном каталоге и заменить прямоугольники с прямыми углами на прямоугольники с закруглёнными углами, и наоборот. Прямоугольник с прямыми углами в SVG-файле задаётся тегом rect, имеющим вид
    \begin{center}
        \centering
        <rect x="50" y="20" width="150" height="150" />
    \end{center}
    Чтобы «закруглить» углы, достаточно добавить атрибуты rx и ry, задающие радиусы закругления. Например,
    \begin{center}
        \centering
        <rect x="50" y="20" width="150" height="150" rx="20" ry="20" />
    \end{center}
    Работоспособность программы нужно проверить на наборе SVG-файлов, загруженных из интернета или созданных в векторном графическом редакторе.

    \section{Решение}

    \subsection{Код}

    \begin{code}
        \inputminted{cpp}{../src/main.cc}
        \caption{main.cc}
    \end{code}

    \subsection{Скриншоты}

    \begin{figure}[!htbp]
        \centering
        \includegraphics[width=\textwidth]{before}
        \caption{До запуска}
    \end{figure}

    \begin{figure}[!htbp]
        \centering
        \includegraphics[width=\textwidth]{after}
        \caption{После запуска}
    \end{figure}

\end{document}
