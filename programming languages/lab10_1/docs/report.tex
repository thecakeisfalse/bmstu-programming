\documentclass{../../iu9lab}

\labnumber{10-1}
\course{Языки и методы программирования}
\title{Реализация итераторов на языке C++}
\group{ИУ9-21Б}
\teacher{Посевин Д. П.}

\begin{document}
    \maketitle
    
    \section{Цель}

    Данная работа предназначена для приобретения навыков разработки контейнерных классов с итераторам.

    \section{Персональный вариант}

    Строка с однонаправленным итератором по словам (слова в строке разделены произвольным количеством пробелов). Должно быть предусмотрено обращение к отдельным символам строки с помощью перегруженной операции <<[ ]>>, работающей за константное время.

    \section{Решение}

    \subsection{Код}

    \begin{code}
        \inputminted{cpp}{../src/main.cc}
        \caption{main.cc}
    \end{code}

    \subsection{Скриншоты}

    \begin{figure}[!htbp]
        \centering
        \includegraphics[width=\textwidth]{terminal}
        \caption{Пример работы программы}
    \end{figure}

\end{document}
