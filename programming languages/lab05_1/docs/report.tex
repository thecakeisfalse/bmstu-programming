\documentclass{../../iu9lab}

\labnumber{5-1}
\course{Языки и методы программирования}
\title{Монады в языке Java}
\group{ИУ9-21Б}
\teacher{Посевин Д. П.}

\begin{document}
    \maketitle
    
    \section{Цель}

    Приобретение навыков использования монад Optional и Stream в программах на языке Java.

    \section{Персональный вариант}

    Таблица, отображающая названия товаров в их цены и количаства имеющихся на складе единиц товара, с операциями:
    \begin{enumerate}
        \item порождение потока названий товаров, стоимость запасов которых превышает указанную сумму денег;
        \item поиск товара, количество единиц которого на складе превышает суммарное количество единиц всех остальных товаров.
    \end{enumerate}
    Проверить работу первой операции нужно путём группировки названий товаров по первой букве названия.

    \section{Решение}

    \subsection{Код}

    \begin{code}
        \inputminted{java}{../src/Test.java}
        \caption{Test.java}
    \end{code}

    \subsection{Скриншоты}

    \begin{figure}[!htbp]
        \centering
        \includegraphics[width=\textwidth]{terminal}
        \caption{Пример работы программы}
    \end{figure}

\end{document}
