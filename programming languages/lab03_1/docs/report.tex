\documentclass{../../iu9lab}

\labnumber{3-1}
\course{Языки и методы программирования}
\title{Полиморфизм на основе интерфейсов в языке Java}
\group{ИУ9-21Б}
\teacher{Посевин Д. П.}

\begin{document}
    \maketitle
    
    \section{Цель}

    Приобретение навыков реализации интерфейсов для обеспечения возможности полиморфной обработки объектов класса.

    \section{Персональный вариант}

    Класс ломаных линий на плоскости с порядком на основе количества пересечений ломаной линии с осями координат.

    \section{Решение}

    \subsection{Код}

    \begin{code}
        \inputminted{java}{../src/Test.java}
        \caption{Test.java}
    \end{code}

    \pagebreak

    \begin{code}
        \inputminted{java}{../src/BrokenLine.java}
        \caption{BrokenLine.java}
    \end{code}

    \begin{code}
        \inputminted{java}{../src/Point.java}
        \caption{Point.java}
    \end{code}

    \subsection{Скриншоты}

    \begin{figure}[ht!]
        \centering        
        \includegraphics[width=\textwidth]{terminal}
        \caption{Пример работы}
    \end{figure}

    \begin{figure}[ht!]
        \centering
        \includegraphics[width=\textwidth]{intellij1}
        \includegraphics[width=\textwidth]{intellij3}
        \caption{Исходный код ч. 1}
    \end{figure}

    \begin{figure}[ht!]
        \includegraphics[width=\textwidth]{intellij2}
        \caption{Исходный код ч. 2}
    \end{figure}

\end{document}